%%%%%%%%%%%%%%%%%%%%%%%%%%%%%%%%%%%%%%%%%%%%%%%%%%%%%%%%%%%%%%%%%%%%%%%%%%%%%
%$HeadURL$
%$LastChangedDate$
%$LastChangedBy$
%$LastChangedRevision$
%%%%%%%%%%%%%%%%%%%%%%%%%%%%%%%%%%%%%%%%%%%%%%%%%%%%%%%%%%%%%%%%%%%%%%%%%%%%%
%
% To compile this document do the following:
%
%   pdflatex simson-svn-guide
%   bibtex simson-svn-guide
%   pdflatex simson-svn-guide
%   pdflatex simson-svn-guide
%
%%%%%%%%%%%%%%%%%%%%%%%%%%%%%%%%%%%%%%%%%%%%%%%%%%%%%%%%%%%%%%%%%%%%%%%%%%%%%

\documentclass[10pt,a4paper]{simson}
\usepackage[swedish,english]{babel}
\usepackage{graphicx}
\usepackage{amsmath}
\usepackage{natbib}
\usepackage[latin1]{inputenc}
\usepackage[colorlinks=true, pdfstartview=FitV, linkcolor=blue, citecolor=blue, urlcolor=blue, pdfkeywords={DNS, Simson, subversion, svn}, pdfauthor={}, pdftitle={User Guide. Produced from $Rev$}]{hyperref}

% Set graphics path
\graphicspath{{./figures/}}

\title{\textbf{SIMSON}\\ParaView Guide\\ $Rev$}

\author{}

\kthno{}
\kthcoverpicture{figures/freestream_front}
\kthinnerbackone{The front page image is a visualization of laminar-turbulent transition induced by ambient free-stream turbulence convected above a flat plate, \emph{i.e.}\ so-called bypass transition \citep{Brandt-Schlatter-Henningson:2004}. The large-eddy simulation used to generate the image \citep{schlatter_brandt_henningson_abstract_2006} is performed with \esoft{Simson} using the ADM-RT subgrid-scale model \citep{schlatter_stolz_kleiser_2004}, further postprocessed using the program \esoft{lambda2} and rendered using \esoft{OpenDX}. Low and high speed streaks are visualized with blue and red isocontours, respectively; green and yellow isocontours indicate the $\lambda_2$ vortex identification criterion \citep{jeong_hussain_1995}. The flow is from lower left to upper right.}
\kthinnerbacktwo{This user guide is compiled from $Rev$.}

\begin{document}

\maketitle

\tableofcontents*


% ####################################################################
\chapter{Introduction}
% ####################################################################
\setcounter{page}{1}\pagenumbering{arabic} To interpret the large
amount of data that can be generated from time dependent, highly
resolved, computer simulation of different flow configurations
visualization is a very important tool. To aid animated visualization
a set python scripts have been created that makes use of VTK \cite{}
and ParaView Open Source projects to generate a sequence of frames
that can be combined into a movie.


% ####################################################################
\chapter{Getting started}
% ####################################################################
In order to use ParaView and VTK the output files from Simson have to
be converted to \efile{.vtk} format. This can be done with the tool
\esoft{lambda2}. This tool also have support to reduce the amount of
data necessary to focus only on the particular part one is interest
in. Once the \efile{.vtk} files have been generated two parameter
files have to be created, one movie*.dat and at least one scene*.dat
file. The movie file defines the global parameters for the movie to be
generated. The movie file then refers to one or more scene files that
define the trajectory and what data that should be loaded.

Once these files have been generated the driver tool is run which
first generates a trajectory file based on the scene files. Once it is
generated it is analyzed so that the driver script knows when it has
to reload data. This is in order to optimize the call to the single
frame script that does the actual image rendering. There is an outer
loop in the sdriver script that iterates over all chunks of the
trajectory file where data is changed. The inner loop in the single
frame script steps through all of trajectory items in a single chunk.


% ====================================================================
\section{The movie.dat file}
% ====================================================================
To configure an animation at least two parameter files are required:
1. The main parameter file holds global parameters for input and
output purposes.
2. Each animation is built up out of scenes where each scene
is controlled through a parameter file that controls what datasets
should be loaded as well as camera movements.

The movie.dat file holds all the global parameters for the movie
generation. After it has been analyzed a stripped down parameter file
that is read by the single\_frame file is generated.

\begin{table}
\begin{verbatim}
# Simson ParaView movie file version
1.0
# Title
Hairpin LES
# Input field data directory
/scratch/mattias/Hairpin/Data-non-uniform-grid/
# Scene files
scene_hairpin_transient_01.dat scene_hairpin_transient_02.dat
# Single frame file
single_frame_hairpin_contour.py
# Trajectory and global movie output parameter filenames
traj_transient.dat glob_transient.dat
# Picture name
frame_hairpin_transient
# Parallel Projection [0 = perspective, 1 = parallel]
0
# Down sample [sx sy sz]
1 1 1
# Cut off [xmin xmax ymin ymax zmin zmax]
0 7000 0 185 0 383
# Background color [RGB]
0 0 0
# View size [xdim ydim]
1024 768
# Background mesh(es) [filenames or "none" if not used]
grid-xz-hairpin.vtk grid-xy-hairpin.vtk
# Mesh color [RGB]
0 30 0
# Start image number
0
\end{verbatim}
\caption{The movie parameter file...}
\end{table}


% --------------------------------------------------------------------
\subsection{The scene.dat file}
% --------------------------------------------------------------------

\begin{verbatim}

\end{verbatim}

% ====================================================================
\section{...}
% ====================================================================

% --------------------------------------------------------------------
\subsection{...}
% --------------------------------------------------------------------


% --------------------------------------------------------------------
\subsection{...}
% --------------------------------------------------------------------


% ####################################################################
\chapter{Examples}
% ####################################################################

\newpage

\bibliographystyle{simson}
\bibliography{fluids}

\end{document}
